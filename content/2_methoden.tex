\section{Methoden}\label{sec:meth}
Im folgenden Kapitel sollen der Rahmen und die wissenschaftlichen Methoden vorgestellt werden, die zur Erreichung des Forschungsziels angewendet wurden. Das Ziel der Ausarbeitung ist die Untersuchung der in Kapitel \ref{sec:fz} vorgestellten Hypothesen, welche aus bereits vorhandenen Forschungsarbeiten extrahiert wurden. Im Sinne des kritischen Rationalismus von Popper, soll versucht werden, eine der vorgestellten Hypothesen zu widerlegen und somit zu falsifizieren \cite[S. 36f]{Doring.2016}. Bei dem vorliegenden Forschungsvorhaben handelt es sich um eine explanative Problemstellung, welche über ein quantitatives Forschungsvorgehen untersucht werden soll.\cite[S. 149f]{Doring.2016}

Die quantitativen Daten sollen über ein Laborexperiment erhoben werden, welches auf einem standardisierten Test basiert. Das Experiment in einer Laborumgebung wurde gewählt um für alle Probanden die gleichen Bedingungen zu schaffen. Da das Ziel der Vergleich unterschiedlicher Lernmethoden ist, spielen äußere Einflüsse und Faktoren eine große Rolle, da sie die Aufmerksamkeit und Konzentration des Probanden erheblich beeinflussen können. Das Laborexperiment zeichnet sich durch ein hohen Kontrollgrad aus und es kann für jeden Testkandidaten eine identische Lernumgebung bezüglich Lichtverhältnisse, Geräuschkulisse und Einrichtung geschaffen werden. Somit ist eine beliebige Anzahl an Versuchswiederholungen unter gleichbleibenden Bedingungen möglich. \cite[S. 206]{Doring.2016}

Die Probanden wurden über einen Münzwurf in zwei gleich große Gruppen aufgeteilt. \textbf{FEHLT STICHPROBENGRÖßE!!!!FEHLT STICHPROBENGRÖßE!!!!FEHLT STICHPROBENGRÖßE!!!!FEHLT STICHPROBENGRÖßE!!!!FEHLT STICHPROBENGRÖßE!!!!}