\section{Fazit und Ausblick}\label{sec:fazit}
Grundlage der Arbeit waren qualitative Forschungsarbeiten, die auf Basis der Loci-Methode einen positiven Effekt bei der Aufnahme von Wissen, unter Anwendung der Augmented Reality Technologie, vermutet haben. Mit Hilfe eines an den LGT-3-Test angelehnten Experiments galt es diese Korrelation zu untersuchen. Hierzu wurde eine Nullhypothese aufgestellt, die es folgend  zu falsifizieren galt. Nach Auswahl einer geeigneten Stichprobe und Durchführung der Test, konnte in der Evaluation der Versuchsergebnisse keine ausreichend signifikante Ausprägung der Korrelation erkannt werden. Folgend ist es nicht möglich die Nullhypothese zu widerlegen und somit die Alternativhypothese als vorläufig gültig zu erklären.

Die Augmented Reality Technologie und die eingesetzten Head-Mounted-Displays sind zum heutigen Zeitpunkt noch relativ unausgereift. Durch die große Anzahl an Einsatzgebieten und den aktuellen Hype werden in der Zukunft noch erhebliche Weiterentwicklung erwartet. Durch diese Entwicklungen werden Probleme die heutzutage noch mit der Technologie bestehen verbessert und sie wird für die breite Masse zugänglich.  

Einer der Kritikpunkte, den Probanden aufgeführt haben, war das geringe Sichtfeld der HoloLens. Dieses ist derzeit auf 35 Grad beschränkt. Microsoft arbeitet allerdings derzeit an einer neuen Technologie, welche eine Sichtfelderweiterung auf 70 Grad ermöglicht. Zudem wurde auf der Messe CES 2019 ein neues AR-HMD von der chinesischen Firma Realmax vorgestellt. Dieses trägt den Namen \textit{Qian} und bietet schon jetzt ein Sichtfeld von 100 Grad. 

Ob das Sichtfeld einen Einfluss auf den Lernerfolg hat, gilt es in weiteren Forschungsarbeiten zu überprüfen. Auch die Positionierung der und Präsentation der Lerninhalte bietet noch viel Material für zukünftige Experimente. So wäre es denkbar farbliche Differenzierungen vorzunehmen, zusätzliches Bild- und Videomaterial einzubinden oder bewegte Objekte und interaktive Elemente zu nutzen um die Inhalte zu präsentieren. Eine weitere Chance, welche der alltägliche Einsatz eines AR-HMDs mit sich bringen würde, ist durch Ebbinghaus vorgestellte 