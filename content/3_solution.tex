\section{Hauptteil}\label{sec:main}
In folgendem Abschnitt soll sowohl der Prozess genauer erläutert werden, der für zu Erstellung des Laborversuchs angewendet wurde. Außerdem soll der Versuch selbst genauer beschrieben und vorgestellt werden. Der Versuchserstellungsprozess und die im Kapitel vorgestellten Aktivitäten sind in Abbildung \ref{fig:ve_proz} zu sehen.

\begin{figure}[h]
\includegraphics[width=0.85\textwidth]{content/versuchserstellung.png}
\centering
\caption{Versuchserstellungsprozess}
\label{fig:ve_proz}
\end{figure}

\subsection{Testauswahl} \label{lgt}
Wie in Kapitel \ref{sec:gedaechtnis} genauer beschrieben wird von der Mehrfach-Speicher-Theorie ausgegangen. Demnach kann das Gedächtnis in vier unterschiedliche Arten Kategorisiert werden. Für den folgenden Versuch wurde sich für das Intermediäre Gedächtnis, also das Alltagsgedächtnis entschieden, welches den Übergang zum Permanenz Gedächtnis darstellt. Es ist zuständig für die Reproduktionsdauer von einer Minute bis mehreren Stunden.

Auf Basis dieser Entscheidung wurde nun eine Recherche durchgeführt um einen wissenschaftlich anerkannten standardisierten Test zu finden, welcher auf die gewählte Gedächtnisart fokussiert ist. Als Ergebnis wurde sich für den Lern-und-Gedächtnistest-3 entschieden (LGT-3). Dieser Test wird seit vielen Jahren angewendet und zählt unter den bekanntesten in dem Gebiet in Deutschland.

Der LGT-3 ist eine Profilbatterie, die aus sechs verschiedenen Tests besteht, die bei vollständiger Durchführung eine Dauer von einer Stunde beanspruchen. Das Ziel der Test ist die Erfassung der raschen Auffassungsgabe, Einprägungsfähigkeit und Intelligenzfunktionen im engeren Sinne mit Hilfe von Paper-Pencil-Verfahren. Die enthaltenen Subtests sind:\cite{Baumler.1974}

\begin{itemize}
    \item Stadtplan
    \item Türkisch
    \item Gegenstände
    \item Bau
    \item Firmenzeichen
    \item Telefonnummern
\end{itemize}

Als Grundlage für den in dieser Arbeit zu erstellenden Versuch wurde der Test \textit{Türkisch} gewählt. Der Subtest besteht aus insgesamt 20 Wortpaaren, die sich aus deutschen und zugehörigen türkischen Übersetzungen zusammensetzen. Die Probanden werden aufgefordert sich so viele Wortpaare zu merken und diese später anhand einer Mehrfachwahl aus 5 gegebenen türkischen Möglichkeiten die passende Übersetzung des deutschen Wortes zu reproduzieren. Anhand der korrekt wiedergegebenen Wortpaare kann anschließend eine Punktzahl vergeben werden.

Um den Test für die Versuchsgruppe B in der Augmented Reality zu realisieren, wurde die Applikation \textit{Type in Space Pro} aus dem Microsoft Store gewählt. Über diese App konnte eine Anbringung der Wortpaare mit Hilfe der Augmented Reality im Versuchslabor stattfinden. Dabei wurde explizit darauf geachtet, dass keine Präsentation von Inhalten an Decken und Fußböden stattfindet. Es wurde eine gleichmäßige und zufällige Verteilung gewählt, die sich auf die Seitenwände beschränkte. Die Mindesthöhe lag bei 0,5m und die maximale Anbringungshöhe bei 2,5m. Somit soll gewährleistet werden, dass den Probanden die Wortpaare so auffällig wie möglich erscheinen und so wenig Lernobjekte wie möglich durch das geringe Sichtfeld übersehen werden.

Um weitere beeinflussende Faktoren beim Vergleich zwischen AR und Papier Lernphase auszuschließen, wurde darauf geachtet, die Inhalte so unauffällig und ähnlich zur Darstellung auf dem Papier wie möglich zu präsentieren. Somit fiel die Wahl bei der Schriftart auf die serifienlose Font Arial. Als Schriffarbe wurde für alle Lerninhalte weiß gewählt. 

\subsection{Pretest und Redesign}
Im Prozessschritt \textit{Pretest} wurde nun eine kleine Gruppe bestehend aus insgesamt 5 Testprobanden hinzugezogen. Diese wurden per Münzwurf in die Testgruppe A für den Paper-Pencil Test und B für die Lernphase mit Hilfe des AR-HMDs eingeteilt. Hierbei wurde darauf geachtet, dass 3 Probanden in der Gruppe für das Lernen mit der AR eingeteilt wurden.

Zunächst wurde eine Lernphase von 1 Minute für beide Gruppe mit einer anschließenden Wartezeit von 1 Minute durchgeführt. Anschließend kam die Rekonstruktionsphase, welche mit 4 Minuten veranschlagt wurde. Dieses Einteilung wird vom LGT-3 vorgeschlagen. 

An den Ergebnissen der Testgruppe B konnte erkannt werden, dass die Ergebnisse erheblich schlechter waren als in der Kontrollgruppe A, die wie erwartet bei 9 und 11 Punkten lagen. Bei anschließenden Interviews mit der AR Lerngruppe konnte festgestellt werden, dass eine Zeit von 1 Minute zu gering war um sich in der AR zurechtzufinden und die Wortpaare zu lernen.

Aufgrund dieser Aussagen im Interview wurde die Lernzeit für beide Gruppen Verdoppelt und es wurde eine Eingewöhnugnsphase an das AR-HMD eingeführt. Anschließend wurde nach dem Redesign ein erneuter Pretest mit 6 Probanden durchgeführt. Hier konnte nun festgestellt werden, dass für beide Testgruppen, wie erwartet, eine Verbesserung der Punktzahl durch die längere Lernphase festzustellen war. Da sich die Ergebnisse jedoch im Bereich von 9 bis 13 befanden, wurde festgelegt, dass keine Erhöhung der Wortpaare stattfinden muss, da sich die Ergebnisse nicht übermäßig vom vorherigen Mittelwert der Punkte unterschieden.

\subsection{Testdurchführung}
Die Durchführung des fertigen Tests bestand aus fünf verschiedenen Phasen. Zunächst wurde der Proband durch einen Münzwurf einer der beiden Testgruppe Zugewiesen. 

Anschließend kam es zu einer Eingewöhnungs- und Erklärungsphase des Testablaufs. Da der LGT-3 als Parallelform vorliegt, war es möglich den Probanden anhand einer abgeänderten psychometrisch gleichwertige Form des Tests sowohl das Lernmaterial, als auch den Antwortbogen zu Verfügung zu stellen. Somit wussten die Probanden direkt, welche Aufgaben sie im Test erwarten  und Fragen konnten im Voraus geklärt werden. Auch in der AR-Lernvariante wurde diese Alternative zur Eingewöhnung zur Verfügung gestellt. Hierbei fand eine gleiche Positionierung der Wortpaare statt, wie sie später im Test vorzufinden sind. Auf diese Weise konnte sichergestellt werden, dass so wenig Lernmaterial wie möglich übersehen wurde und den Probanden war es möglich sich mit der Hololens vertraut zu machen.

Anschließend wurde das Lernmaterial ausgeteilt und es begann eine zwei Minütige Lernphase. In dieser Lernphase wurde der Proband allein im Raum gelassen, um so wenigen Störungen durch äußere Einflüsse wie möglich ausgesetzt zu sein und die Konzentration zu fördern. Nach Ablauf der Zeit wurde das Lernmaterial entzogen und der Proband wurde exakt eine Minute lang über die Erfahrung interviewt, um weiteres Lernen durch Wiederholung im Kopf zu vermeiden. Nach Ablauf der Minute begann die Reproduktionsphase und der Antwortbogen wurde vom Probanden ausgefüllt. Hierzu hatte er genau 4 Minuten zeit. Er wurde außerdem aufgefordert den Antwortbogen vollständig auszufüllen und Antworten die er nicht wusste so gut wie möglich zu erraten.

\subsection{Aufbereitung der Daten}
Nach der Testdurchführung war eine Aufbereitung der Daten notwendig. Hierzu wurde die Erreichte Punktzahl des Probanden ermittelt und anschließend in ein Excel-Dokument übertragen. Weiter wurde eine Liste angelegt, welche die Prozentuale Beantwortung der einzelnen Wortpaare festhielt. Hiermit sollte überwacht werden, ob Wortpaare unterdurchschnittlich schlecht beantwortet werden, um eventuelle Fehler in der Positionierung und der Auffindung dieser festzustellen.

Anschließend fand eine Auswertung der erfassten und aufbereiteten Daten fest. Die Ergebnisse dieser Auswertung sind in Kapitel \ref{sec:evaluation} genauer beschrieben.


