\section{Einleitung}\label{sec:intro}

Im ersten Abschnitt der vorliegenden Arbeit erfolgt die Erläuterung der dem Thema zugrundeliegenden Ausgangssituation und dem Forschungszugang. Im Anschluss wird auf die Forschungsfrage und das sich ergebende Forschungsziel eingegangen.

\subsection{Ausgangssituation}
Der Hype um Virtual Reality und Augmented Reality nimmt laut dem Gardner Hype-Cycle von 2018 langsam ab. Die VR ist im Hype-Cycle von 2018 bereits nicht mehr aufzufinden und zählt als ausgereifte Technologie. Auch die Augmented Reality soll in den nächsten fünf Jahren das \textit{Plateu of Productivity} erreichen und somit schon bald nicht mehr als neue Technologie angesehen werden \cite{Gartner.2018}. Um jedoch das volle Potential der Technologien zu erkennen, gilt es nach dem Hype noch viel Forschung auf dem Gebiet zu betreiben um realistische Einsatzgebiete zu erkennen. 

Die Weiterentwicklung und Evolution von Augmented Reality Headsets kann dazu führen, dass das dauerhafte Tragen eines solchen Augmented-Reality Head Mounted Displays (HMD) im Alltag zur Normalität wird. Somit ist es möglich durch die Einblendung von Wissensinhalten unterbewusst zu lernen und daraus resultierend eine effizientere Nutzung der Zeit zu ermöglichen.


\subsection{Forschungszugang}
\enquote{Denn Wissen selbst ist Macht}. Die deutsche Übersetzung der berühmten Worte von Francis Bacon, die auf das Jahr 1597 zurückzuführen sind, gehören zu einem der bekanntesten Zitate der heutigen Zeit \cite{RodrigezGarcia.2001}. Das zentrale Streben der Menschheit nach Wissen und Macht führt zu einem Drang immer neue Methoden zu finden, welche die Aneignung von Wissen verbessern und beschleunigen sollen.

Eine der bewährten Lernmethoden ist die Loci-Methode. Diese wurde schon im antiken Griechenland angewendet. Sie basiert auf der Verbindung des räumlichen Gedächtnisses und dem zu lernenden Inhalt. Die Anwender dieser Technik bewegen sich durch einen realen oder imaginären Raum und verbinden Inhalte mit spezifischen Punkten und Gegenständen in der Umgebung. Sowohl die räumliche Orientierung als auch die Fähigkeit neue Erinnerungen zu erstellen wird vom Gehirn über den Hippocampus realisiert. Daraus lässt sich schließen, dass eine räumliche Beziehung zu Inhalten das Erinnerungsvermögen an die Inhalte positiv beeinflusst. \cite{OKeefe.1978} \cite{Manns.2009}

Über die sogenannte Augmented-Reality (AR) oder Mixed-Reality (MR) Technologien ist es seit einigen Jahren möglich, virtuelle Objekte in unterschiedliche Räume zu platzieren und es dem Nutzer zu ermöglichen diese zu begehen und zu erfahren. Diese neue Technologie ermöglicht eine Erweiterung der zuvor vorgestellten Loci-Methode und bietet somit eine alt bewährte aber zugleich innovative Art zu lernen. Dem Anwender wird es mit der Hilfe eines Augemented Reality Headset ermöglicht selbstständig Inhalte in seinem Sichtfeld zu platzieren und diese zu modifizieren. Es ist außerdem nicht mehr nur möglich Schrift und Bilder zum Lernen zu nutzen, sondern auch bewegtes Bildmaterial oder Interaktive Elemente in die Umgebung einzufügen.\cite{Schart.2017} 

\subsection{Anverwandte Arbeiten}\label{sec:relatedWork}
Die Nutzung von Augmented Reality im Arbeitsalltag wurde schon in einigen wissenschaftlichen Ausarbeitungen erforscht. Auch die Anwendung im Entertainment Segment und Gesundheitssektor wird viel diskutiert. Ein weiteres Einsatzgebiet, das jedoch noch nicht so umfänglich erforscht wurde, ist die Nutzung von Augmented-Reality HMDs im Bereich der kognitiven Merkfähigkeit. 

Es wurden mit der Hilfe qualitativer Forschungsarbeiten Hypothesen aufgestellt, die einen positiven Einfluss der Augmented-Reality Technologie auf den Lernerfolg behaupten. So wurde  auf dem \textit{ IEEE International Symposium on Mixed and Augemnted Reality} die Arbeit eines Forschungsteams der \textit{Nara Institute of Science and Technology} bestehend aus Yuichiro Fujimoto, Goshiro Yamamoto, Takafumi Taketomi, Jun Miyazaki und Hirokazu Kato Forschungsergebnisse unter dem Titel \textit{Relationship between Features of Augmented Reality and User Memorization} veröffentlicht. Hier wurden 18 Probanden im Alter zwischen 23 und 27 getestet und Interviews geführt. Als Ergebnis der Arbeit wurde eine Hypothese aufgestellt, dass die Visualisierung von Inhalten mit Hilfe von AR HMDs in einem Raum einen signifikanten positiven Effekt auf die Merkfähigkeit darstellt. \cite{Fujimoto.05.11.201208.11.2012}

Zu ähnlichen Forschungsergebnissen kam eine Forschungsgruppe vom Massachusetts Institute of Technology (MIT). Sie haben im Zuge der Arbeit ein Interface mit dem Namen \textit{Nevermind} entworfen. Dieses soll mit Hilfe von Augemented-Reality Technologien den Anwendern beim Lernen von Inhalten über das räumliche Gedächtnis helfen und somit die Effizienz beim Lernen steigern. \cite{Rosello.2016}


\subsection{Forschungsfrage}
Die Technologie der Augmented Reality wird stetig verbessert und in Zukunft werden HMDs in dem Bereich eine immer weitere Verbreitung finden. Diese großflächige Verbreitung und Einbindung in den Alltag birgt viele neue Chancen für Anwender. Es stellt sich nun die Frage ob eine direkte Verbindung zwischen dem Lernen mit einem Augmented-Reality HMD und einem besseren Lernerfolg besteht.


\subsection{Forschungsziel} \label{sec:fz}
Die zuvor genannten Anverwandten Arbeiten und theoretischen Ansätze legen nahe, dass eine Verbindung zwischen dem Lernen über das räumliche Gedächtnis mit der Hilfe eines Augmented Reality HMD und einem besseren Lernerfolg besteht. Das Ziel der Arbeit ist es somit die folgenden aufgestellten Thesen zu überprüfen.

\begin{itemize}
    \item H0: Wenn ein Augmented Reality HMD Lernen von Inhalten eingesetzt wird, dann ist keine oder eine negative Verbindung zum Lernerfolg zu erkennen.\\
    \item H1: Wenn ein Augmented Reality HMD Lernen von Inhalten eingesetzt wird, dann ist eine positive Verbindung zum Lernerfolg zu erkennen.\\
\end{itemize}
