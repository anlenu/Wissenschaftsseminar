\section{Evaluation}\label{sec:evaluation}
Das folgende Kapitel befasst sich mit der Präsentation und der Auswertung der Daten, die im zuvor beschriebenen Laborexperiment erfasst wurden. Hierbei werden zunächst die Ergebnisse der Gruppe A vorgestellt, welche sich mit der Lernmethode anhand des Paper-Pencil Tests befasst hat. Anschließend findet die Präsentation der Resultate der Gruppe B statt, in der die Lernphase mittels Augmented Reality HMD durchgeführt wurde. Abschließend findet ein Vergleich der beiden Gruppen und die daraus resultierende Hypothesenprüfung statt.

\subsection{Ergebnisse Gruppe A}
Die Testgruppe A nutzte die Pencil-Paper Lernmethode, die vom LGT-3 vorgegeben wird, mit der modifizierten Bearbeitungszeit von zwei anstatt einer Minute. Der Test wurde insgesamt mit 33 Probanden durchgeführt und die Ergebnisse sind in Abbildung \ref{fig:erg_paper} zu sehen.

\begin{figure}[h]
\includegraphics[width=\textwidth]{content/erg_paper.png}
\centering
\caption{Ergebnisse der Testgruppe A Pencil-Paper}
\label{fig:erg_paper}
\end{figure}

Der Großteil der Probanden konnte zwischen 9 und 17 Wortpaare in der Reproduktionsphase richtig zuordnen. Lediglich einem Testkandidaten war es nur möglich 6 Punkte zu erreichen. Die Höchste Punktzahl von 19 wurde ebenfalls von nur einem Probanden erreicht. Alle 20 Wortpaare richtig zuzuordnen war keinem der Testkandidaten möglich. Der arithmetische Mittelwert aller Testdurchläufe in der Gruppe A konnte auf 12,9091 festgelegt werden. Dieses Ergebnis zeigt, dass im Durchschnitt weniger als 13 Wortpaare pro Testdurchlauf richtig beantwortet wurden. Die Standardabweichung bei Berücksichtigung aller Testergebnisse liegt bei 3,2438. 


\subsection{Ergebnisse Gruppe B}
Die Ergebnisse der Gruppe B spiegeln die Testresultate der Probanden wider, welche die zwei minütige Lernphase mit Hilfe des Augmented Reality HMDs \textit{Microsoft Hololens} absolviert haben. Wie bei Gruppe A, wurde der Test ebenfalls mit 33 Probanden durchgeführt. Die Ergebnisse, welche durch das Experiment erlangt wurden sind in Abbildung \ref{fig:erg_ar} zu finden.

\begin{figure}[h]
\includegraphics[width=\textwidth]{content/erg_ar.png}
\centering
\caption{Ergebnisse der Testgruppe B Augmented Reality}
\label{fig:erg_ar}
\end{figure}

Bei der Auswertung der Ergebnisse der Gruppe B kann festgestellt werden, dass die meisten Testkandidaten zwischen 10 und 18 Wortpaare korrekt reproduzieren konnten. Der Proband mit der geringsten Punktzahl konnte hierbei mit 7 Punkten abschließen, während der höchste erzielte Wert 19 war, was bedeutet, dass nur ein Wortpaar falsch beantwortet wurde. Die Höchstpunktzahl und somit die korrekte Zuordnung aller türkischen Worte zum passenden deutschen Wort, konnte niemand erreichen. Der arithmetische Mittelwert der Testresultate in Gruppe B liegt bei 13,3333. Dieser Wert sagt aus, dass im Durchschnitt ein Proband über 13 Wortpaare richtig zuordnen konnte. Unter Berücksichtigung aller Testresultate konnte eine Standardabweichung von 3,0686 festgestellt werden.


\subsection{Ergebnisse Monitoring Wortpaare in Gruppe B}
Um Fehler oder schwer auffindbare Wortpaare im erstellten Augmented Reality Lernbereich festzustellen, wurden die Ergebnisse der korrekt und fehlerhaft beantworteten Wortpaare aufgezeichnet. Der Stand des Monitoring nachdem alle 33 Probanden getestet wurden, ist in Abbildung \ref{fig:ar_beantwortung} zu sehen. Unter der Berücksichtigung der Komplexität der Wortpaare, sollte der relative Wert der richtig beantworteten Inhalte nicht zu sehr abweichen und nicht unter einen Grenzwert fallen. Als Grenzwert wurde hierbei im Voraus eine richtige Beantwortung von über 50 Prozent gewählt. Das am häufigsten korrekt beantwortete Wortpaar war hierbei \textit{nein = yok} mit 78,79\% und wurde von den Probanden auch am einprägsamsten, aufgrund seiner einfachen Komplexität, beschrieben. Das am seltensten richtig beantwortete Wortpaar in allen Testdurchläufen der Gruppe B war \textit{Herz = yürek} mit 57,58\%. Da das Wort jedoch eine höhere Komplexität aufweist und noch deutlich über dem Schwellwert von 50\% liegt, wurden hier keine Änderungen an der Positionierung vorgenommen.


\begin{figure}[h]
\includegraphics[width=\textwidth]{content/ar_beantwortung.png}
\centering
\caption{Ergebnisse des Monitoring der Wortpaare in Gruppe B}
\label{fig:ar_beantwortung}
\end{figure}

\subsection{Vergleich der Ergebnisse}
Folgend soll ein Vergleich der erzielten Testresultate von Gruppe A und Gruppe B vollzogen werden.

Der arithmetische Mittelwert ergibt sich aus der Summe aller Testresultate der Gruppe dividiert durch die Anzahl der Probanden. Er sagt aus, wie viele richtige Wortpaare die jeweilige Gruppe durchschnittlich richtig beantworten konnte. Während diejenigen Testkandidaten, welche die Lernphase mittels Augmented Reality durchführten, im arithmetischen Mittel 13,3333 Wortpaare richtig beantworten konnten, war es den Probanden in der Gruppe A lediglich möglich 12,9091 Wortpaare richtig zuzuordnen. Daraus resultiert, dass Gruppe A in diesem Experiment durchschnittlich bessere Ergebnisse erzielen konnte als Gruppe B.

Der Wert der Standardabweichung, welcher auch als Stichprobenstreuung bekannt ist, gibt die durchschnittliche Entfernung der gemessenen Werte zum arithmetischen Mittelwert an. Je größer diese Zahl ist, desto höher ist die Streuung der Werte in der gezogenen Stichprobe. Die Standardabweichung der Gruppe A beträgt 3,2438, während bei Gruppe B eine Streuung von 3,0686 gemessen werden kann. Der Vergleich dieser beiden Werte zeigt, dass die Ergebnisse, der Probanden, welche die Lernphase mit Hilfe der AR-Technologie durchgeführt haben, weniger weit vom Mittelwert abweichen, als die Probanden der Gruppe A. Dies bedeutet, dass die erreichten Punktzahlen der Testkandidaten in Gruppe B weniger voneinander abweichen. Je geringer die Standardabweichung, desto repräsentativer ist der arithmetische Mittelwert.

\subsection{Hypothesenprüfung}
Nach der Durchführung des Experiments, der Präsentation und dem Vergleich der Ergebnisse ist es nun notwendig festzustellen, ob die erhobenen Daten eine ausreichend signifikante Tendenz liefern, um eine der zuvor aufgestellten Thesen zu widerlegen. Dies wurde mit Hilfe eines T-Tests bei unabhängigen Stichproben ermittelt. Um diese Analyse durchzuführen wurde die Software für Statistik- und Analyse \textit{SPSS} von IBM verwendet. 

Die Ergebnisse dieser Analyse waren ein p-Wert von 0,546. Da dieser deutlich über dem gesetzten Signifikanzlevel von 0,1 (10\%) liegt, kann die gewählte Nullhypothese \textit{H0} nicht abgelehnt und somit nicht widerlegt werden. Obwohl eine Tendenz der arithmetischen Mittelwerte zu erkennen ist, welche annehmen lässt, dass eine Lernphase mittels Augmented Reality Technologien einen positiven Effekt hat, sind die Ergebnisse nicht signifikant genug. Da der kritische Wert nicht unterschritten wurde, ist es nicht möglich auszuschließen, ob die ermittelten Testresultate und die zu sehende Abweichung durch Zufall entstanden sind.

