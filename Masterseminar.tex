\documentclass{fhwsjournale}

\usepackage[ngerman]{babel}
\usepackage[utf8]{inputenc}
\usepackage[T1]{fontenc}
\usepackage{subfigure}
\usepackage{graphicx}

\usepackage{lmodern}
\usepackage{amsfonts}
\usepackage{amssymb}
\usepackage{amsmath}
\usepackage[babel,german=quotes]{csquotes}
\usepackage[hidelinks]{hyperref}
%\usepackage{breakurl}
\usepackage{listings}
\usepackage{caption}
\usepackage{nameref}
\usepackage{hyperref}
\usepackage{epstopdf}
\usepackage{siunitx}
\usepackage{placeins}

% for dummy text
\usepackage{blindtext}

%============ PAPERDATEN ================
\title{Auswirkungen für das Lernen durch die Nutzung der Augmented-Reality Technologie}

\fhwsYear{WS 2018}
\newcommand{\university}{Hochschule für angewandte Wissenschaften Würzburg--Schweinfurt}
\newcommand{\modulename}{FHWS -- Master: Informationssysteme, Modul: Wissenschaftsseminar} 
\newcommand{\name}{Andreas Nüchter}

% Kopfzeile
\markboth{}{\modulename}
% Hier steht der Text, der am Ende die Autordaten enthält.           
\author{\name\\\modulename\\\university}
%============ PAPERDATEN ================

\sisetup{locale=DE}

%============ MACROS ================
\newcommand{\qq}[1]{\glqq #1\grqq{}}
\newcommand{\fig}[1]{Abbildung~\ref{#1}}
\newcommand{\floor}[1]{\lfloor #1 \rfloor}
\newcommand{\tinsec}[1]{\SI{#1}{\second}}
\newcommand{\expp}[1]{\exp \left( #1  \right)}
%============ MACROS ================

%============ ABSTRACT ================
\begin{abstract} 



\end{abstract}
%============ ABSTRACT ================

%============ CATEGEORY ================
% WELCHE KATEGORIE HAT DAS PAPER?
% ES GIBT:
%		G = General
%		H = Hardware
%		IT= IT-Organisation
%		S = Software
%		IS= Information Systems
%		Sub=Category is "free"
\category{IS}{Information Systems}{} % TODO
%============ CATEGEORY ================            

%============ TERMS ================                      
% HAUPT-KEYWORDS. HIER STEHEN EIN PAAR BUZZWORDS ZUR THEMATIK
\terms{Augmented Reality, Gedächtnis, Lernmethoden} % TODO 
%============ TERMS ================   

%============ KEYWORDS ================   
% HIER STEHEN ERWEITERTE KEYWORDS.                 
\keywords{Loci-Methode, Wissensaneignung, Head-Mounted-Display, Microsoft Hololens} % TODO
%============ KEYWORDS ================        

\begin{document}

%============ FOOTER ================  
\begin{bottomstuff} 
Autoren Adresse: Andreas Nüchter, Hochschule für angewandte Wissenschaften Würzburg--Schweinfurt, Deutschland, www.fhws.de
\end{bottomstuff}
%============ FOOTER ================  

\maketitle

%-------- KAPTIEL 1 --------------
% HIER SOLLTE DIE EINLEITUNG/HINFÜHRUNG/PROBLEMERÖRTERUNG STEHEN
\section{Einleitung}\label{sec:intro}

Im ersten Abschnitt der vorliegenden Arbeit erfolgt die Erläuterung der dem Thema zugrundeliegenden Ausgangssituation und dem Forschungszugang. Im Anschluss wird auf die Forschungsfrage und das sich ergebende Forschungsziel eingegangen.

\subsection{Ausgangssituation}
Der Hype um Virtual Reality und Augmented Reality nimmt laut dem Gardner Hype-Cycle von 2018 langsam ab. Die VR ist im Hype-Cycle von 2018 bereits nicht mehr aufzufinden und zählt als ausgereifte Technologie. Auch die Augmented Reality soll in den nächsten fünf Jahren das \textit{Plateu of Productivity} erreichen und somit schon bald nicht mehr als neue Technologie angesehen werden \cite{Gartner.2018}. Um jedoch das volle Potential der Technologien zu erkennen, gilt es nach dem Hype noch viel Forschung auf dem Gebiet zu betreiben um realistische Einsatzgebiete zu erkennen. 

Die Weiterentwicklung und Evolution von Augmented Reality Headsets kann dazu führen, dass das dauerhafte Tragen eines solchen Augmented-Reality Head Mounted Displays (HMD) im Alltag zur Normalität wird. Somit ist es möglich durch die Einblendung von Wissensinhalten unterbewusst zu lernen und daraus resultierend eine effizientere Nutzung der Zeit zu ermöglichen.


\subsection{Forschungszugang}
\enquote{Denn Wissen selbst ist Macht}. Die deutsche Übersetzung der berühmten Worte von Francis Bacon, die auf das Jahr 1597 zurückzuführen sind, gehören zu einem der bekanntesten Zitate der heutigen Zeit \cite{RodrigezGarcia.2001}. Das zentrale Streben der Menschheit nach Wissen und Macht führt zu einem Drang immer neue Methoden zu finden, welche die Aneignung von Wissen verbessern und beschleunigen sollen.

Eine der bewährten Lernmethoden ist die Loci-Methode. Diese wurde schon im antiken Griechenland angewendet. Sie basiert auf der Verbindung des räumlichen Gedächtnisses und dem zu lernenden Inhalt. Die Anwender dieser Technik bewegen sich durch einen realen oder imaginären Raum und verbinden Inhalte mit spezifischen Punkten und Gegenständen in der Umgebung. Sowohl die räumliche Orientierung als auch die Fähigkeit neue Erinnerungen zu erstellen wird vom Gehirn über den Hippocampus realisiert. Daraus lässt sich schließen, dass eine räumliche Beziehung zu Inhalten das Erinnerungsvermögen an die Inhalte positiv beeinflusst. \cite{OKeefe.1978} \cite{Manns.2009}

Über die sogenannte Augmented-Reality (AR) oder Mixed-Reality (MR) Technologien ist es seit einigen Jahren möglich, virtuelle Objekte in unterschiedliche Räume zu platzieren und es dem Nutzer zu ermöglichen diese zu begehen und zu erfahren. Diese neue Technologie ermöglicht eine Erweiterung der zuvor vorgestellten Loci-Methode und bietet somit eine alt bewährte aber zugleich innovative Art zu lernen. Dem Anwender wird es mit der Hilfe eines Augemented Reality Headset ermöglicht selbstständig Inhalte in seinem Sichtfeld zu platzieren und diese zu modifizieren. Es ist außerdem nicht mehr nur möglich Schrift und Bilder zum Lernen zu nutzen, sondern auch bewegtes Bildmaterial oder Interaktive Elemente in die Umgebung einzufügen.\cite{Schart.2017} 

\subsection{Anverwandte Arbeiten}\label{sec:relatedWork}
Die Nutzung von Augmented Reality im Arbeitsalltag wurde schon in einigen wissenschaftlichen Ausarbeitungen erforscht. Auch die Anwendung im Entertainment Segment und Gesundheitssektor wird viel diskutiert. Ein weiteres Einsatzgebiet, das jedoch noch nicht so umfänglich erforscht wurde, ist die Nutzung von Augmented-Reality HMDs im Bereich der kognitiven Merkfähigkeit. 

Es wurden mit der Hilfe qualitativer Forschungsarbeiten Hypothesen aufgestellt, die einen positiven Einfluss der Augmented-Reality Technologie auf den Lernerfolg behaupten. So wurde  auf dem \textit{ IEEE International Symposium on Mixed and Augemnted Reality} die Arbeit eines Forschungsteams der \textit{Nara Institute of Science and Technology} bestehend aus Yuichiro Fujimoto, Goshiro Yamamoto, Takafumi Taketomi, Jun Miyazaki und Hirokazu Kato Forschungsergebnisse unter dem Titel \textit{Relationship between Features of Augmented Reality and User Memorization} veröffentlicht. Hier wurden 18 Probanden im Alter zwischen 23 und 27 getestet und Interviews geführt. Als Ergebnis der Arbeit wurde eine Hypothese aufgestellt, dass die Visualisierung von Inhalten mit Hilfe von AR HMDs in einem Raum einen signifikanten positiven Effekt auf die Merkfähigkeit darstellt. \cite{Fujimoto.05.11.201208.11.2012}

Zu ähnlichen Forschungsergebnissen kam eine Forschungsgruppe vom Massachusetts Institute of Technology (MIT). Sie haben im Zuge der Arbeit ein Interface mit dem Namen \textit{Nevermind} entworfen. Dieses soll mit Hilfe von Augemented-Reality Technologien den Anwendern beim Lernen von Inhalten über das räumliche Gedächtnis helfen und somit die Effizienz beim Lernen steigern. \cite{Rosello.2016}


\subsection{Forschungsfrage}
Die Technologie der Augmented Reality wird stetig verbessert und in Zukunft werden HMDs in dem Bereich eine immer weitere Verbreitung finden. Diese großflächige Verbreitung und Einbindung in den Alltag birgt viele neue Chancen für Anwender. Es stellt sich nun die Frage ob eine direkte Verbindung zwischen dem Lernen mit einem Augmented-Reality HMD und einem besseren Lernerfolg besteht.


\subsection{Forschungsziel} \label{sec:fz}
Die zuvor genannten Anverwandten Arbeiten und theoretischen Ansätze legen nahe, dass eine Verbindung zwischen dem Lernen über das räumliche Gedächtnis mit der Hilfe eines Augmented Reality HMD und einem besseren Lernerfolg besteht. Das Ziel der Arbeit ist es somit die folgenden aufgestellten Thesen zu überprüfen.

\begin{itemize}
    \item H0: Wenn ein Augmented Reality HMD Lernen von Inhalten eingesetzt wird, dann ist keine oder eine negative Verbindung zum Lernerfolg zu erkennen.\\
    \item H1: Wenn ein Augmented Reality HMD Lernen von Inhalten eingesetzt wird, dann ist eine positive Verbindung zum Lernerfolg zu erkennen.\\
\end{itemize}

%-------- KAPTIEL 1 --------------

%-------- KAPTIEL 2 --------------
% WELCHE ANDEREN ANSÄTZE WURDEN IN DER THEMATIK BEREITS DURCH ANDERE TEAMS ODER PERSONEN UNTERNOMMEN - WAS GIBT ES FÜR ALTERNATIVE ANSÄTZE.
\section{Grundlagen}\label{sec:gdl}
Im folgenden Kapitel sollen einige theoretische Grundlagen geschaffen werden, die als Rahmenbedingungen zum Verständnis der weiteren Arbeit benötigt werden. Zunächst soll die Augmented Reality Technologie kurz erklärt und anschließend die für die Ausarbeitung wichtigsten Informationen über die Funktion des Gedächtnisses gegeben werden.

\subsection{Augmented Reality}
Die am meisten zitierte Definition für Augmented Reality, oder Erweiterte Realität, stammt von Ronald T. Azuma aus dem Artikel \textit{A Survey of Augmented Reality}. Sie definiert den Begriff über drei grundsätzliche Aussagen:
\begin{itemize}
    \item In der Augmented Reality wird die Realität und die virtuelle Realität miteinander kombiniert und zum Teil überlagert.\\
    \item Die Interaktivität in der Augmented Reality findet in Echtzeit statt.\\
    \item Es besteht ein 3-Dimensionaler Bezug zwischen Realen und Virtuellen Objekten. Somit sind 2 Dimensionale Einblendungen ins Sichtfeld, wie Head-up-Displays keine Augmented Reality.\\
\end{itemize}

Diese Grundsätze zur Definition von Augmented Reality von Azuma lässt im Gegensatz zu alternativen Definitionen auch Anwendungen ohne ein Head-Mounted-Display zu. \cite{RonaldT.Azuma.1997} 

Um dieses Vorhaben der Augmented Reality umzusetzen werden unterschiedliche Komponenten benötigt. Zunächst werden als Hardware mehrere Kameras und Sensoren zur Aufnahme und Rekonstruktion der 3-Dimensionalen Umgebung eingesetzt. Weiter wird eine Tracking-Software und ein sogenannter Renderer genutzt um eine Anzeige und ordnungsgemäße Überlagerung von Gegenständen zu ermöglichen und die gewünschten Funktionen auszuführen. Um dem Nutzer die neu geschaffene, erweiterte Realität zu präsentieren, ist letztendlich noch ein Anzeigegerät, wie ein Head-mounted-Display oder ein Monitor, erforderlich. Siehe Abbildung \ref{fig:AR_Komp}. \cite[S. 45f]{Schart.2017}

\begin{figure}[h]
\includegraphics[width=\textwidth]{content/GrafikAR.png}
\centering
\caption{Komponenten von Augmented Reality. In Anlehnung an \protect \cite{Schilling.2008}}
\label{fig:AR_Komp}
\end{figure}

Die genau Erfassung, Aufbereitung und Präsentation der Sensordaten ist für eine gute User-Experience in der AR von hoher Bedeutung. Für das sogenannten Tracking werden zum heutigen Zeitpunkt unterschiedliche Technologien eingesetzt. Es wird unterschieden in visuelles und optisches Tracking. Das optische Tracking erfolgt mit Hilfe von Licht und Infrarot. Die Informationen werden über mehrere Kameras aufgenommen und sind heutzutage das gebräuchlichste Mittel zur Analyse der Umgebung. Visuelles Tracking hingegen erfolgt über sogenannte Marker. Diese künstlichen Symbole werden in der realen Welt angebracht und dienen als Orientierungspunkt für die Bilderkennungsmethoden, die den gelieferten Videostream analysieren. \cite[S.47f]{Schart.2017}

Eine Erweiterung dieser Tracking Verfahren, die zum Beispiel bei der Microsoft HoloLens zum Einsatz kommt, ist das sogenannte SLAM-Tracking. Das Akronym SLAM steht für \textit{Simultaneous Localization and Mapping}. Es basiert auf einer Karte, die entweder fest vorgegeben oder durch die zuvor beschriebenen Tracking-Methoden zunächst erstellt wird. Anhand dieser Karte und der relativen Position zu den erfassten Objekten, kann eine Position des Geräts und somit die Position des Anwenders ermittelt werden. \cite[S.49]{Schart.2017}

Probleme, welche das Tracking und die Lokalisierung für die Augmented Reality derzeit noch erschweren, sind vor Allem äußere Einflüsse. Herumlaufende Menschen und sich bewegende Gegenstände, sowie sich stark ändernde Lichtverhältnisse, führen häufig zu Problemen. Aufgrund dessen ist ein Einsatz meist nur in geschlossenen Räumen möglich. Die Sonneneinstrahlung im Freien und zu viele sich bewegende Objekte stören das Tracking und führen zu schlechten Ergebnissen.+

Die in dieser Ausarbeitung verwendete AR-Technologie war das HMD Hololens von Microsoft. Diese verfügt über zwei HD Light-Engines, die für die Hologramme in den See-Through-Lenses zuständig sind und besitzt 2.3M total light points. Die Hololens verfügt über eine geringe Latenz und hat ein Sichtfeld von 35 Grad. Das Betriebssystem basiert auf Windows 10 und ist über Netzwerk mit einem externen Device verknüpfbar. So ist es theoretisch möglich dem Probanden bei der Versuchsdurchführung zuzusehen. \cite{Microsoft.2018}


\subsection{Funktionsweise des menschlichen Gedächtnisses} \label{sec:gedaechtnis}
Zum Verständnis der Ausarbeitung und des Versuchsaufbaus sind ebenfalls einige Grundlagen im Bezug auf die Funktionsweise des menschlichen Gedächtnisses notwendig. Da das menschliche Gehirn, Erinnerungen und das Gedächtnis noch bei weitem nicht vollständig erforscht wurde, wird von den bisher bekannten Annahmen ausgegangen. Diese sollen folgend zusammengefasst werden, um eine gemeinsame Grundlage zu schaffen. 

\begin{figure}[h]
\includegraphics[width=\textwidth]{content/Gedaechtnis_Arten.png}
\centering
\caption{Gedächtnisarten nach Bäumler 1974}
\label{fig:gd_arten}
\end{figure}

Es existiert der Ansatz, dass das menschliche Gedächtnis in mehrere differenzierte Bereiche eingeteilt werden kann. Dieser Erklärungsversuch wird auch als \textit{Mehrfach-Speicher-Theorie des Gedächtnisses} bezeichnet. Günther Bäumler \citeN{Baumler.1974}, einer maßgeblichen Autoren des \textit{Lern-und-Gedächtnistest LGT-3} auf den in Kapitel \ref{lgt} genauer eingegangen wird, benennt hierbei vier Unterteilungen des Gedächtnisses. Diese sind in Abbildung \ref{fig:gd_arten} zu sehen. Folgend werden die einzelnen Bereiche kurz erläutert.

\subsubsection{ikonisches Gedächtnis}
Das \textit{ikonische Gedächtnis} ist auch als \textit{Ultrakurzzeitgedächtnis} oder \textit{Sensorisches Gedächtnis} bekannt. Es wird häufig als Nachflackern bezeichnet. In diesem Abschnitt werden Informationen nur wenige Hundertstel-Sekunden gespeichert bis zu maximal 1 oder 2 Sekunden. Es dient nur der kurzen Verarbeitung von Information. Ein Chemischer Vorgang findet hierbei nicht statt, sondern es wird rein über elektrische Impulse realisiert.

\subsubsection{Präsenzgedächtnis}
Als nächste Instanz wird das \textit{Präsenzgedächtnis} genannt. Das auch als \textit{Kurzzeitgedächtnis} bezeichnete Arbeitsgedächtnis, dient zum Behalten von Informationen von 1 bis zu 30 Sekunden. Es dient zum Beispiel dem Verständnis von langen Sätzen, Merken von Telefonnummer und ist dem \textit{RAM} eines Computers am ähnlichsten. Die Halbwertszeit des Gedächtnisses wurde auf ungefähr 20 Sekunden bestimmt.

\subsubsection{intermediäres Gedächtnis}
Werden Informationen nach 1 Minute bis zu mehreren Stunden Abgerufen, weißt dies auf den Zugriff auf das \textit{intermediäre Gedächtnis} hin. Dieses wird auch als \textit{mittelfristiges} oder \textit{Alltags Gedächtnis} bezeichnet. Es befindet sich zwischen dem Kurzzeitgedächtnis und dem Langzeitgedächtnis und bezeichnet den Weg vom Arbeitsgedächtnis hin zum längerfristigen Wissen. Es wird deshalb auch als Alltagsgedächtnis bezeichnet.

\subsubsection{Permanenz Gedächtnis}
Das \textit{Permanenz Gedächtnis} ist schwer genau vom intermediären Gedächtnis zu trennen. Die untere Grenze in der Literatur variiert häufig. Bäumler legt sich hierbei aber auf eine untere Grenze von 6 Stunden fest und auf eine obere Grenze von mehreren Monaten bis Jahrzehnten. Das intermediäre Gedächtnis und das Permanenz Gedächtnis werden in anderen Unterteilungen häufig zusammengefasst als \textit{Langzeitgedächtnis} \cite[S. 2]{Gruber.2018}. In das Permanenz Gedächtnis werden nur Informationen aufgenommen, die für besonders wichtig erachtet werden und durch Wiederholung trainiert werden.

\subsubsection{Vergessenskurve}
Eine weitere Theorie, die das Zusammenspiel der Gedächtnisarten und die Funktionsweise der Informationsspeicherung im Gehirn verdeutlicht, ist die \textit{Kurve des Vergessens} nach Prof. Dr. Herrmann Ebbinghaus \citeN{Ebbinghaus.1885}. Er untersuchte, wie lange sinnlose Silben im Gedächtnis abrufbar sind, nachdem Sie zu 100 Prozent auswendig gelernt hatte und reproduzieren konnte. Aus diesem Versuch ergab sich die in Abbildung \ref{fig:vgkurve} zu sehende Kurve. Diese wird bis zum heutigen Zeitpunkt noch als gültig betrachtet. Wie in der Abbildung zu erkennen ist, ist in der ersten Stunde der größte Gedächtnisverlust zu verzeichnen. \cite[S. 3]{Gruber.2018}

\begin{figure}[h]
\includegraphics[width=0.5\textwidth]{content/Vergessenkurve.png}
\centering
\caption{Kurve des Vergessens nach \protect \cite{Ebbinghaus.1885}}
\label{fig:vgkurve}
\end{figure}

In Abbildung \ref{fig:wdh_kurve} ist ebenfalls die Vergessenskurve zu sehen. Jedoch wurden hier zusätzliche Wiederholungen des Lernstoffs in einem bestimmten zeitlichen Abstand durchgeführlt. Wie zu erkennen ist, wird die Kurve, welche die richtig wiedergegebenen Silben repräsentiert, mit häufigeren Wiederholungen immer flacher. Dies bedeutet, dass mit häufigeren Wiederholungen der Lernaufwand und der richtig wiedergegebene Inhalt stetig ansteigt und weniger gelerntes erneut vergessen wird. Häufige Wiederholungen ermöglichen somit einen besseren Zugang ins Langzeitgedächtnis.\cite[S. 4]{Gruber.2018}

\begin{figure}[h]
\includegraphics[width=0.5\textwidth]{content/wiederholung.png}
\centering
\caption{Kurve des Vergessens mit Wiederholungen nach \protect \cite{Ebbinghaus.1885}}
\label{fig:wdh_kurve}
\end{figure}
\vspace{-0.5mm} % hack damit Haupteil nicht vor der ersten subsection umgebrochen wird
%-------- KAPTIEL 2 --------------

%-------- KAPTIEL 3 --------------
% IN KAPITEL 3 SOLLTE MAN SICH MIT DEN MATERIALIEN UND METHODEN AUSEINANDERSETZEN - WIE HABE ICH MEINE LÖSUNG GEFUNDEN - WISSENSCHAFTLICHE METHODEN... 
\section{Methoden}\label{sec:meth}
Im folgenden Kapitel sollen der Rahmen und die wissenschaftlichen Methoden vorgestellt werden, die zur Erreichung des Forschungsziels angewendet wurden. Das Ziel der Ausarbeitung ist die Untersuchung der in Kapitel \ref{sec:fz} vorgestellten Hypothesen, welche aus bereits vorhandenen Forschungsarbeiten extrahiert wurden. Im Sinne des kritischen Rationalismus von Popper, soll versucht werden, eine der vorgestellten Hypothesen zu widerlegen und somit zu falsifizieren \cite[S. 36f]{Doring.2016}. Bei dem vorliegenden Forschungsvorhaben handelt es sich um eine explanative Problemstellung, welche über ein quantitatives Forschungsvorgehen untersucht werden soll.\cite[S. 149f]{Doring.2016}

Die quantitativen Daten sollen über ein Laborexperiment erhoben werden, welches auf einem standardisierten Test basiert. Das Experiment in einer Laborumgebung wurde gewählt um für alle Probanden die gleichen Bedingungen zu schaffen. Da das Ziel der Vergleich unterschiedlicher Lernmethoden ist, spielen äußere Einflüsse und Faktoren eine große Rolle, da sie die Aufmerksamkeit und Konzentration des Probanden erheblich beeinflussen können. Das Laborexperiment zeichnet sich durch ein hohen Kontrollgrad aus und es kann für jeden Testkandidaten eine identische Lernumgebung bezüglich Lichtverhältnisse, Geräuschkulisse und Einrichtung geschaffen werden. Somit ist eine beliebige Anzahl an Versuchswiederholungen unter gleichbleibenden Bedingungen möglich. \cite[S. 206]{Doring.2016}

Die Probanden wurden über einen Münzwurf in zwei gleich große Gruppen aufgeteilt. \textbf{FEHLT STICHPROBENGRÖßE!!!!FEHLT STICHPROBENGRÖßE!!!!FEHLT STICHPROBENGRÖßE!!!!FEHLT STICHPROBENGRÖßE!!!!FEHLT STICHPROBENGRÖßE!!!!}
%-------- KAPTIEL 3 --------------

%-------- KAPTIEL 4 --------------
% IN KAPTEL 4 SOLLTET IHR EURE LÖSUNG VORSTELLEN. QUASI DER HAUPTTEIL.
\section{Hauptteil}\label{sec:main}
In folgendem Abschnitt soll sowohl der Prozess genauer erläutert werden, der für zu Erstellung des Laborversuchs angewendet wurde. Außerdem soll der Versuch selbst genauer beschrieben und vorgestellt werden. Der Versuchserstellungsprozess und die im Kapitel vorgestellten Aktivitäten sind in Abbildung \ref{fig:ve_proz} zu sehen.

\begin{figure}[h]
\includegraphics[width=0.85\textwidth]{content/versuchserstellung.png}
\centering
\caption{Versuchserstellungsprozess}
\label{fig:ve_proz}
\end{figure}

\subsection{Testauswahl} \label{lgt}
Wie in Kapitel \ref{sec:gedaechtnis} genauer beschrieben wird von der Mehrfach-Speicher-Theorie ausgegangen. Demnach kann das Gedächtnis in vier unterschiedliche Arten Kategorisiert werden. Für den folgenden Versuch wurde sich für das Intermediäre Gedächtnis, also das Alltagsgedächtnis entschieden, welches den Übergang zum Permanenz Gedächtnis darstellt. Es ist zuständig für die Reproduktionsdauer von einer Minute bis mehreren Stunden.

Auf Basis dieser Entscheidung wurde nun eine Recherche durchgeführt um einen wissenschaftlich anerkannten standardisierten Test zu finden, welcher auf die gewählte Gedächtnisart fokussiert ist. Als Ergebnis wurde sich für den Lern-und-Gedächtnistest-3 entschieden (LGT-3). Dieser Test wird seit vielen Jahren angewendet und zählt unter den bekanntesten in dem Gebiet in Deutschland.

Der LGT-3 ist eine Profilbatterie, die aus sechs verschiedenen Tests besteht, die bei vollständiger Durchführung eine Dauer von einer Stunde beanspruchen. Das Ziel der Test ist die Erfassung der raschen Auffassungsgabe, Einprägungsfähigkeit und Intelligenzfunktionen im engeren Sinne mit Hilfe von Paper-Pencil-Verfahren. Die enthaltenen Subtests sind:\cite{Baumler.1974}

\begin{itemize}
    \item Stadtplan
    \item Türkisch
    \item Gegenstände
    \item Bau
    \item Firmenzeichen
    \item Telefonnummern
\end{itemize}

Als Grundlage für den in dieser Arbeit zu erstellenden Versuch wurde der Test \textit{Türkisch} gewählt. Der Subtest besteht aus insgesamt 20 Wortpaaren, die sich aus deutschen und zugehörigen türkischen Übersetzungen zusammensetzen. Die Probanden werden aufgefordert sich so viele Wortpaare zu merken und diese später anhand einer Mehrfachwahl aus 5 gegebenen türkischen Möglichkeiten die passende Übersetzung des deutschen Wortes zu reproduzieren. Anhand der korrekt wiedergegebenen Wortpaare kann anschließend eine Punktzahl vergeben werden.

Um den Test für die Versuchsgruppe B in der Augmented Reality zu realisieren, wurde die Applikation \textit{Type in Space Pro} aus dem Microsoft Store gewählt. Über diese App konnte eine Anbringung der Wortpaare mit Hilfe der Augmented Reality im Versuchslabor stattfinden. Dabei wurde explizit darauf geachtet, dass keine Präsentation von Inhalten an Decken und Fußböden stattfindet. Es wurde eine gleichmäßige und zufällige Verteilung gewählt, die sich auf die Seitenwände beschränkte. Die Mindesthöhe lag bei 0,5m und die maximale Anbringungshöhe bei 2,5m. Somit soll gewährleistet werden, dass den Probanden die Wortpaare so auffällig wie möglich erscheinen und so wenig Lernobjekte wie möglich durch das geringe Sichtfeld übersehen werden.

Um weitere beeinflussende Faktoren beim Vergleich zwischen AR und Papier Lernphase auszuschließen, wurde darauf geachtet, die Inhalte so unauffällig und ähnlich zur Darstellung auf dem Papier wie möglich zu präsentieren. Somit fiel die Wahl bei der Schriftart auf die serifienlose Font Arial. Als Schriffarbe wurde für alle Lerninhalte weiß gewählt. 

\subsection{Pretest und Redesign}
Im Prozessschritt \textit{Pretest} wurde nun eine kleine Gruppe bestehend aus insgesamt 5 Testprobanden hinzugezogen. Diese wurden per Münzwurf in die Testgruppe A für den Paper-Pencil Test und B für die Lernphase mit Hilfe des AR-HMDs eingeteilt. Hierbei wurde darauf geachtet, dass 3 Probanden in der Gruppe für das Lernen mit der AR eingeteilt wurden.

Zunächst wurde eine Lernphase von 1 Minute für beide Gruppe mit einer anschließenden Wartezeit von 1 Minute durchgeführt. Anschließend kam die Rekonstruktionsphase, welche mit 4 Minuten veranschlagt wurde. Dieses Einteilung wird vom LGT-3 vorgeschlagen. 

An den Ergebnissen der Testgruppe B konnte erkannt werden, dass die Ergebnisse erheblich schlechter waren als in der Kontrollgruppe A, die wie erwartet bei 9 und 11 Punkten lagen. Bei anschließenden Interviews mit der AR Lerngruppe konnte festgestellt werden, dass eine Zeit von 1 Minute zu gering war um sich in der AR zurechtzufinden und die Wortpaare zu lernen.

Aufgrund dieser Aussagen im Interview wurde die Lernzeit für beide Gruppen Verdoppelt und es wurde eine Eingewöhnugnsphase an das AR-HMD eingeführt. Anschließend wurde nach dem Redesign ein erneuter Pretest mit 6 Probanden durchgeführt. Hier konnte nun festgestellt werden, dass für beide Testgruppen, wie erwartet, eine Verbesserung der Punktzahl durch die längere Lernphase festzustellen war. Da sich die Ergebnisse jedoch im Bereich von 9 bis 13 befanden, wurde festgelegt, dass keine Erhöhung der Wortpaare stattfinden muss, da sich die Ergebnisse nicht übermäßig vom vorherigen Mittelwert der Punkte unterschieden.

\subsection{Testdurchführung}
Die Durchführung des fertigen Tests bestand aus fünf verschiedenen Phasen. Zunächst wurde der Proband durch einen Münzwurf einer der beiden Testgruppe Zugewiesen. 

Anschließend kam es zu einer Eingewöhnungs- und Erklärungsphase des Testablaufs. Da der LGT-3 als Parallelform vorliegt, war es möglich den Probanden anhand einer abgeänderten psychometrisch gleichwertige Form des Tests sowohl das Lernmaterial, als auch den Antwortbogen zu Verfügung zu stellen. Somit wussten die Probanden direkt, welche Aufgaben sie im Test erwarten  und Fragen konnten im Voraus geklärt werden. Auch in der AR-Lernvariante wurde diese Alternative zur Eingewöhnung zur Verfügung gestellt. Hierbei fand eine gleiche Positionierung der Wortpaare statt, wie sie später im Test vorzufinden sind. Auf diese Weise konnte sichergestellt werden, dass so wenig Lernmaterial wie möglich übersehen wurde und den Probanden war es möglich sich mit der Hololens vertraut zu machen.

Anschließend wurde das Lernmaterial ausgeteilt und es begann eine zwei Minütige Lernphase. In dieser Lernphase wurde der Proband allein im Raum gelassen, um so wenigen Störungen durch äußere Einflüsse wie möglich ausgesetzt zu sein und die Konzentration zu fördern. Nach Ablauf der Zeit wurde das Lernmaterial entzogen und der Proband wurde exakt eine Minute lang über die Erfahrung interviewt, um weiteres Lernen durch Wiederholung im Kopf zu vermeiden. Nach Ablauf der Minute begann die Reproduktionsphase und der Antwortbogen wurde vom Probanden ausgefüllt. Hierzu hatte er genau 4 Minuten zeit. Er wurde außerdem aufgefordert den Antwortbogen vollständig auszufüllen und Antworten die er nicht wusste so gut wie möglich zu erraten.

\subsection{Aufbereitung der Daten}
Nach der Testdurchführung war eine Aufbereitung der Daten notwendig. Hierzu wurde die Erreichte Punktzahl des Probanden ermittelt und anschließend in ein Excel-Dokument übertragen. Weiter wurde eine Liste angelegt, welche die Prozentuale Beantwortung der einzelnen Wortpaare festhielt. Hiermit sollte überwacht werden, ob Wortpaare unterdurchschnittlich schlecht beantwortet werden, um eventuelle Fehler in der Positionierung und der Auffindung dieser festzustellen.

Anschließend fand eine Auswertung der erfassten und aufbereiteten Daten fest. Die Ergebnisse dieser Auswertung sind in Kapitel \ref{sec:evaluation} genauer beschrieben.



%-------- KAPTIEL 4 --------------

%-------- KAPTIEL 5 --------------
% KAPITEL 5 BEHANDELT EINE ART EVALUATION. HIERBEI GILT ES EURE ARBEIT IN RELATION ZU SETZEN ZU ANDEREN ANSÄTZEN ODER GENERELL ZU EVALUIEREN. ES SOLLTE AUCH SEHR KRITISCH DISKUTIERT WERDEN.
\section{Evaluation}\label{sec:evaluation}
Das folgende Kapitel befasst sich mit der Präsentation und der Auswertung der Daten, die im zuvor beschriebenen Laborexperiment erfasst wurden. Hierbei werden zunächst die Ergebnisse der Gruppe A vorgestellt, welche sich mit der Lernmethode anhand des Paper-Pencil Tests befasst hat. Anschließend findet die Präsentation der Resultate der Gruppe B statt, in der die Lernphase mittels Augmented Reality HMD durchgeführt wurde. Abschließend findet ein Vergleich der beiden Gruppen und die daraus resultierende Hypothesenprüfung statt.

\subsection{Ergebnisse Gruppe A}
Die Testgruppe A nutzte die Pencil-Paper Lernmethode, die vom LGT-3 vorgegeben wird, mit der modifizierten Bearbeitungszeit von zwei anstatt einer Minute. Der Test wurde insgesamt mit 33 Probanden durchgeführt und die Ergebnisse sind in Abbildung \ref{fig:erg_paper} zu sehen.

\begin{figure}[h]
\includegraphics[width=\textwidth]{content/erg_paper.png}
\centering
\caption{Ergebnisse der Testgruppe A Pencil-Paper}
\label{fig:erg_paper}
\end{figure}

Der Großteil der Probanden konnte zwischen 9 und 17 Wortpaare in der Reproduktionsphase richtig zuordnen. Lediglich einem Testkandidaten war es nur möglich 6 Punkte zu erreichen. Die Höchste Punktzahl von 19 wurde ebenfalls von nur einem Probanden erreicht. Alle 20 Wortpaare richtig zuzuordnen war keinem der Testkandidaten möglich. Der arithmetische Mittelwert aller Testdurchläufe in der Gruppe A konnte auf 12,9091 festgelegt werden. Dieses Ergebnis zeigt, dass im Durchschnitt weniger als 13 Wortpaare pro Testdurchlauf richtig beantwortet wurden. Die Standardabweichung bei Berücksichtigung aller Testergebnisse liegt bei 3,2438. 


\subsection{Ergebnisse Gruppe B}
Die Ergebnisse der Gruppe B spiegeln die Testresultate der Probanden wider, welche die zwei minütige Lernphase mit Hilfe des Augmented Reality HMDs \textit{Microsoft Hololens} absolviert haben. Wie bei Gruppe A, wurde der Test ebenfalls mit 33 Probanden durchgeführt. Die Ergebnisse, welche durch das Experiment erlangt wurden sind in Abbildung \ref{fig:erg_ar} zu finden.

\begin{figure}[h]
\includegraphics[width=\textwidth]{content/erg_ar.png}
\centering
\caption{Ergebnisse der Testgruppe B Augmented Reality}
\label{fig:erg_ar}
\end{figure}

Bei der Auswertung der Ergebnisse der Gruppe B kann festgestellt werden, dass die meisten Testkandidaten zwischen 10 und 18 Wortpaare korrekt reproduzieren konnten. Der Proband mit der geringsten Punktzahl konnte hierbei mit 7 Punkten abschließen, während der höchste erzielte Wert 19 war, was bedeutet, dass nur ein Wortpaar falsch beantwortet wurde. Die Höchstpunktzahl und somit die korrekte Zuordnung aller türkischen Worte zum passenden deutschen Wort, konnte niemand erreichen. Der arithmetische Mittelwert der Testresultate in Gruppe B liegt bei 13,3333. Dieser Wert sagt aus, dass im Durchschnitt ein Proband über 13 Wortpaare richtig zuordnen konnte. Unter Berücksichtigung aller Testresultate konnte eine Standardabweichung von 3,0686 festgestellt werden.


\subsection{Ergebnisse Monitoring Wortpaare in Gruppe B}
Um Fehler oder schwer auffindbare Wortpaare im erstellten Augmented Reality Lernbereich festzustellen, wurden die Ergebnisse der korrekt und fehlerhaft beantworteten Wortpaare aufgezeichnet. Der Stand des Monitoring nachdem alle 33 Probanden getestet wurden, ist in Abbildung \ref{fig:ar_beantwortung} zu sehen. Unter der Berücksichtigung der Komplexität der Wortpaare, sollte der relative Wert der richtig beantworteten Inhalte nicht zu sehr abweichen und nicht unter einen Grenzwert fallen. Als Grenzwert wurde hierbei im Voraus eine richtige Beantwortung von über 50 Prozent gewählt. Das am häufigsten korrekt beantwortete Wortpaar war hierbei \textit{nein = yok} mit 78,79\% und wurde von den Probanden auch am einprägsamsten, aufgrund seiner einfachen Komplexität, beschrieben. Das am seltensten richtig beantwortete Wortpaar in allen Testdurchläufen der Gruppe B war \textit{Herz = yürek} mit 57,58\%. Da das Wort jedoch eine höhere Komplexität aufweist und noch deutlich über dem Schwellwert von 50\% liegt, wurden hier keine Änderungen an der Positionierung vorgenommen.


\begin{figure}[h]
\includegraphics[width=\textwidth]{content/ar_beantwortung.png}
\centering
\caption{Ergebnisse des Monitoring der Wortpaare in Gruppe B}
\label{fig:ar_beantwortung}
\end{figure}

\subsection{Vergleich der Ergebnisse}
Folgend soll ein Vergleich der erzielten Testresultate von Gruppe A und Gruppe B vollzogen werden.

Der arithmetische Mittelwert ergibt sich aus der Summe aller Testresultate der Gruppe dividiert durch die Anzahl der Probanden. Er sagt aus, wie viele richtige Wortpaare die jeweilige Gruppe durchschnittlich richtig beantworten konnte. Während diejenigen Testkandidaten, welche die Lernphase mittels Augmented Reality durchführten, im arithmetischen Mittel 13,3333 Wortpaare richtig beantworten konnten, war es den Probanden in der Gruppe A lediglich möglich 12,9091 Wortpaare richtig zuzuordnen. Daraus resultiert, dass Gruppe A in diesem Experiment durchschnittlich bessere Ergebnisse erzielen konnte als Gruppe B.

Der Wert der Standardabweichung, welcher auch als Stichprobenstreuung bekannt ist, gibt die durchschnittliche Entfernung der gemessenen Werte zum arithmetischen Mittelwert an. Je größer diese Zahl ist, desto höher ist die Streuung der Werte in der gezogenen Stichprobe. Die Standardabweichung der Gruppe A beträgt 3,2438, während bei Gruppe B eine Streuung von 3,0686 gemessen werden kann. Der Vergleich dieser beiden Werte zeigt, dass die Ergebnisse, der Probanden, welche die Lernphase mit Hilfe der AR-Technologie durchgeführt haben, weniger weit vom Mittelwert abweichen, als die Probanden der Gruppe A. Dies bedeutet, dass die erreichten Punktzahlen der Testkandidaten in Gruppe B weniger voneinander abweichen. Je geringer die Standardabweichung, desto repräsentativer ist der arithmetische Mittelwert.

\subsection{Hypothesenprüfung}
Nach der Durchführung des Experiments, der Präsentation und dem Vergleich der Ergebnisse ist es nun notwendig festzustellen, ob die erhobenen Daten eine ausreichend signifikante Tendenz liefern, um eine der zuvor aufgestellten Thesen zu widerlegen. Dies wurde mit Hilfe eines T-Tests bei unabhängigen Stichproben ermittelt. Um diese Analyse durchzuführen wurde die Software für Statistik- und Analyse \textit{SPSS} von IBM verwendet. 

Die Ergebnisse dieser Analyse waren ein p-Wert von 0,546. Da dieser deutlich über dem gesetzten Signifikanzlevel von 0,1 (10\%) liegt, kann die gewählte Nullhypothese \textit{H0} nicht abgelehnt und somit nicht widerlegt werden. Obwohl eine Tendenz der arithmetischen Mittelwerte zu erkennen ist, welche annehmen lässt, dass eine Lernphase mittels Augmented Reality Technologien einen positiven Effekt hat, sind die Ergebnisse nicht signifikant genug. Da der kritische Wert nicht unterschritten wurde, ist es nicht möglich auszuschließen, ob die ermittelten Testresultate und die zu sehende Abweichung durch Zufall entstanden sind.


%-------- KAPTIEL 5 --------------

%-------- KAPTIEL 6 --------------
% KAPITEL 6 BILDET DEN SCHLUSS - HIER SOLL EIN FAZIT/SCHLUßFOLGERUNG/AUSBLICK BESCHRIEBEN WERDEN.
\section{Fazit und Ausblick}\label{sec:fazit}
Grundlage der Arbeit waren qualitative Forschungsarbeiten, die auf Basis der Loci-Methode einen positiven Effekt bei der Aufnahme von Wissen, unter Anwendung der Augmented Reality Technologie, vermutet haben. Mit Hilfe eines an den LGT-3-Test angelehnten Experiments galt es diese Korrelation zu untersuchen. Hierzu wurde eine Nullhypothese aufgestellt, die es folgend  zu falsifizieren galt. Nach Auswahl einer geeigneten Stichprobe und Durchführung der Test, konnte in der Evaluation der Versuchsergebnisse keine ausreichend signifikante Ausprägung der Korrelation erkannt werden. Folgend ist es nicht möglich die Nullhypothese zu widerlegen und somit die Alternativhypothese als vorläufig gültig zu erklären.

Die Augmented Reality Technologie und die eingesetzten Head-Mounted-Displays sind zum heutigen Zeitpunkt noch relativ unausgereift. Durch die große Anzahl an Einsatzgebieten und den aktuellen Hype werden in der Zukunft noch erhebliche Weiterentwicklung erwartet. Durch diese Entwicklungen werden Probleme die heutzutage noch mit der Technologie bestehen verbessert und sie wird für die breite Masse zugänglich.  

Einer der Kritikpunkte, den Probanden aufgeführt haben, war das geringe Sichtfeld der HoloLens. Dieses ist derzeit auf 35 Grad beschränkt. Microsoft arbeitet allerdings derzeit an einer neuen Technologie, welche eine Sichtfelderweiterung auf 70 Grad ermöglicht. Zudem wurde auf der Messe CES 2019 ein neues AR-HMD von der chinesischen Firma Realmax vorgestellt. Dieses trägt den Namen \textit{Qian} und bietet schon jetzt ein Sichtfeld von 100 Grad. 

Ob das Sichtfeld einen Einfluss auf den Lernerfolg hat, gilt es in weiteren Forschungsarbeiten zu überprüfen. Auch die Positionierung der und Präsentation der Lerninhalte bietet noch viel Material für zukünftige Experimente. So wäre es denkbar farbliche Differenzierungen vorzunehmen, zusätzliches Bild- und Videomaterial einzubinden oder bewegte Objekte und interaktive Elemente zu nutzen um die Inhalte zu präsentieren. Eine weitere Chance, welche der alltägliche Einsatz eines AR-HMDs mit sich bringen würde, ist durch Ebbinghaus vorgestellte 
%-------- KAPTIEL 6 --------------

\newpage
\bibliography{fhwsbib}
\bibliographystyle{fhwsjournale}

% AKTUELL STEHT HIER NUR DER RECEIVED PART. DIESER BESCHREIBT BEI EINEM PAPER DEN ABSTAND ZWISCHEN ERSTEINREICHUNG AUF EINER KONFERENZ UND DEM MONAT/JAHR DER ANNAME => JE GERINGER DIESER ABSTAND UM SO AKTUELLER UND GGF. BESSER IST DAS PAPER
\received{01/2019}{00/0000}{00/0000} 

\end{document}
